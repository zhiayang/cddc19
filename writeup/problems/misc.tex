% polyglot, very serious, fancy numbers

\section{Polyglot}

	\subsection{Solution}

		The obvious solution was to put each sentence into Google Translate\footnote{\url{https://translate.google.com}};
		all 10 sentences were some variation of \enquote{the first letter of this language is the flag}. Coupled with
		the input format, given as \ttt{[01][02]\~[03][04][05][06]\&[07][08][09][10]!}, we were able to decode the
		flag.

		\begin{nicetable}[1.3][0.9\textwidth]{ X[.5,c,m] | X[c,m] | X[3,c,m] }
			Number  &   Language    &   Sentence                      \\ \hline
			\ttt{01}&   Hindi       &   इस भाषा का पहला चरित्र झंडा बनाता है। \\
			\ttt{02}&   Indonesian  &   Karakter pertama bahasa ini yang mengibarkan bendera.\\
			\ttt{03}&   Chinese     &   这种语言的第一个字符构成了旗帜。\\
			\ttt{04}&   Dutch       &   Het eerste teken van deze taal vormt de vlag. \\
			\ttt{05}&   Danish      &   Det første tegn på dette sprog udgør flag. \\
			\ttt{06}&   Catalan     &   El primer caràcter d’aquest idioma constitueix la bandera. \\
			\ttt{07}&   Norwegian   &   Det første tegnet av dette språket utgjør flagget. \\
			\ttt{08}&   Spanish     &   El primer carácter de este lenguaje lo constituye la bandera. \\
			\ttt{09}&   Hmong       &   Thawj qhov cim ntawm hom lus no ua rau tus chij. \\
			\ttt{10}&   Croatian    &   Prvi znak ovog jezika čini zastavu.
		\end{nicetable}

		Assembled, the message was \ttt{HI\~CDDC\&NSHC}; it helped that there was a coherent message to verify that
		Google Translate didn't misdetect any of the langauges.

	% end subsection

	\subsection{Flag}
		The flag for this challenge was \cddcflag{HI\~{}CDDC\&NSHC}.
	% end subsection

% end section
go s
