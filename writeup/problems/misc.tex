% polyglot, very serious, fancy numbers

% ufgh
\newfontscript{Devanagari}{deva,dev2}
\newfontface{\hindi}[Script=Devanagari]{Lohit-Devanagari}

\section{Polyglot}

	What to do, when presented with languages that we can't understand?

	\subsection{Solution}

		The obvious solution was to put each sentence into Google Translate\footnote{\url{https://translate.google.com}};
		all 10 sentences were some variation of \enquote{the first letter of this language is the flag}. Coupled with
		the input format, given as \ttt{[01][02]\~[03][04][05][06]\&[07][08][09][10]!}, we were able to decode the
		flag.

		\begin{nicetable}[1.3][0.95\textwidth]{ X[.5,c,m] | X[.7,c,m] | X[3.5,c,m] }
			Number  &   Language    &   Sentence                      \\ \hline
			\ttt{01}&   Hindi       &   {\hindi इस भाषा का पहला चरित्र झंडा बनाता है। }\\
			\ttt{02}&   Indonesian  &   Karakter pertama bahasa ini yang mengibarkan bendera.\\
			\ttt{03}&   Chinese     &   这种语言的第一个字符构成了旗帜。\\
			\ttt{04}&   Dutch       &   Het eerste teken van deze taal vormt de vlag. \\
			\ttt{05}&   Danish      &   Det første tegn på dette sprog udgør flag. \\
			\ttt{06}&   Catalan     &   El primer caràcter d'aquest idioma constitueix la bandera. \\
			\ttt{07}&   Norwegian   &   Det første tegnet av dette språket utgjør flagget. \\
			\ttt{08}&   Spanish     &   El primer carácter de este lenguaje lo constituye la bandera. \\
			\ttt{09}&   Hmong       &   Thawj qhov cim ntawm hom lus no ua rau tus chij. \\
			\ttt{10}&   Croatian    &   Prvi znak ovog jezika čini zastavu.
		\end{nicetable}

		Assembled, the message was \ttt{HI\~{}CDDC\&NSHC}; it helped that there was a coherent message to verify that
		Google Translate didn't misdetect any of the langauges.

	% end subsection

	\subsection{Flag}
		The flag for this challenge was \cddcflag{HI\~{}CDDC\&NSHC}.
	% end subsection

% end section


\pagebreak
\section{Do You Fancy Numbers?}

	We are presented with a picture of what are apparently numbers. On first guess they appear to be Chinese in nature.

	\subsection{Solution}

		Indeed, a quick trawl of Wikipedia led us to Suzhou numerals\footnote{\url{https://en.wikipedia.org/wiki/Suzhou_numerals}}, which
		contained everything we needed to decode the message.

		After substituting them for arabic numerals, we get the following string of numbers:
		\ttt{36} \ttt{67} \ttt{68} \ttt{68} \ttt{67} \ttt{49} \ttt{57} \ttt{36} \ttt{123} \ttt{53} \ttt{48} \ttt{95} \ttt{121} \ttt{48}
		\ttt{117} \ttt{95} \ttt{102} \ttt{52} \ttt{78} \ttt{99} \ttt{89} \ttt{95} \ttt{102} \ttt{108} \ttt{48} \ttt{87} \ttt{51} \ttt{114}
		\ttt{95} \ttt{78} \ttt{117} \ttt{77} \ttt{98} \ttt{51} \ttt{82} \ttt{53} \ttt{125}.

		Putting them through an ASCII decoder yields the following Base64 encoded string:
		\ttt{JCBDIEQgRCB\\DIDEgOSAkIHsgNSAwIF8geSAwIHUgXyBmIDQgTiBjIFkgXyBmIGwgMCBXIDMgciBfIE4gdSBNIGIgMy\\BSIDUgfQ==}

		Finally, putting that through a Base64 decoder yields the flag (with some spaces).

	% end subsection

	\subsection{Flag}
		The flag for this challenge was \cddcflag{50\_y0u\_f4NcY\_fl0W3r\_NuMb3R5}.
	% end subsection

% end section


\pagebreak
\section{Unzip}

	Oh, it isn't a zip bomb, is it?

	\subsection{Solution}
		Running \ttt{zip2john} on the file revealed a password protected file \ttt{flag.png}, and that the encryption
		format was \ttt{pkzip}.

		\begin{listing}[!htbp]
			\begin{minted}[autogobble,xleftmargin=0.075\textwidth,xrightmargin=0.075\textwidth,frame=leftline,framesep=4mm,framerule=0.4mm]{sh}
				$ zip2john Un.Zip
				Created directory: /root/.john
				ver 1.0 Un.Zip/flag.png PKZIP Encr: cmplen=579, decmplen=579, crc=C5C3E066
				Un.Zip/flag.png:$pkzip...pkzip2$:flag.png:Un.Zip::Un.Zip
			\end{minted}
		\end{listing}

		We could then use \ttt{pkcrack}\footnote{\url{https://www.unix-ag.uni-kl.de/~conrad/krypto/pkcrack.html}} to
		crack the file:

		\begin{listing}[!htbp]
			\begin{minted}[autogobble,xleftmargin=0.075\textwidth,xrightmargin=0.075\textwidth,frame=leftline,framesep=4mm,framerule=0.4mm]{sh}
				$ ./extract -p Un.Zip flag.png
			\end{minted}
		\end{listing}

		This reveals the flag:

		\begin{figure}[!htbp]\centering
			\includegraphics[width=60mm]{figures/unzip.png} \vspace{5mm}
			\caption{Pixels!}
		\end{figure}

	% end subsection

	\subsection{Flag}
		The flag for this challenge was \cddcflag{zZziIipPp}.
	% end subsection

% end section










