% count1, count2, count4.

\section{Count 1: Baby}

	This was a fairly simple code golfing challenge; inspection of the provided \ttt{count1-baby.py} file
	gave the character limit as \num{53} characters. The starting code is given as this:

	\begin{listing}[!htbp]
		\begin{ccode}
			#include <stdio.h>

			int main(int argc, char *argv[])
			{
				int i;

				for( i = 1 ; i < 10000 ; i++ )
				{
					printf("%d,", i);
				}

				return 0;
			}
		\end{ccode}
	\end{listing}

	\subsection{Solution}

		Compiling and running the program shows that the expected output of the minified program are the numbers \ttt{1} to \ttt{9999},
		separated by commas (with a trailing comma after \ttt{9999}).

		Besides the obvious steps like removing whitespace and newlines, knowledge of the C standard and ignorance of compiler warnings
		can yield the following transformations:

		\begin{bulletlist}
			& \ttt{printf} is an implicitly defined standard library function; \ttt{stdio.h} can be removed
			& Functions do not require a type specifier; \ttt{int main} can be shortened to \ttt{main}
			& \ttt{main} does not need to take arguments, leaving \ttt{main()}
			& The verifier rejects spaces, so \ttt{int i} has to be changed to \ttt{int(i)} (which is valid)
			& \ttt{main} does not need to explicitly \ttt{return 0}
		\end{bulletlist}

		\pagebreak
		This yields the final program, at \num{53} characters long:

		\begin{listing}[!htbp]\begin{ccode}
			main(){int(i);for(i=1;i<10000;i++){printf("%d,",i);}}
			\end{ccode}
			\caption{The solution for Count 1: Baby}
			\label{code:count1}
		\end{listing}

	% end subsection

	\subsection{Flag}

		The flag for this challenge was \cddcflag{Count2\_is\_waiting\_Please\_enjoy!}.

	% end subsection

% end section



\pagebreak
\section{Count 2: Wildness}

	The required output for this challenge is the same as the previous challenge (Count 1: Baby), but the restrictions have been
	tightened. This time, OP's friend \enquote{doesn't like to go wild}, or, in other words, the following characters cannot be
	used in the source code: \ttt{<== w1lD ==>}. Additionally, the character limit has been reduced to \num{41}.

	\subsection{Solution}

		Starting from the code for Count 1 (Listing \ref{code:count1}), we can apply some more non-obvious tricks:

		\begin{bulletlist}
			& Variables don't need a type specifier, and will default to \ttt{int}
			& Variables with static storage duration will be initialised to \ttt{0} (obviating the need for \ttt{=})
			& \ttt{!=} is equivalent to a bitwise XOR (\ttt{\^{}}) for integers (removing \ttt{<})
			& Taking advantage of \ttt{i++} semantics can replace \ttt{1e4} with \ttt{9999} (since \ttt{1} can't be used)
		\end{bulletlist}

		This yields the final program, again at exactly \num{41} characters:

		\begin{listing}[!htbp]\begin{ccode}
			i;main(){for(;i++^9999;printf("%d,",i));}
			\end{ccode}
			\caption{The solution for Count 2: Wildness}
		\end{listing}

	% end subsection

	\subsection{Flag}

		The flag for this challenge was \cddcflag{This\_really\_helps\_m3\_a\_lot}

	% end subsection

% end section



\pagebreak
\section{Count 4: Madness - Filter}

	For this challenge, the character limit is slightly relaxed to \num{44} characters, but OP's friend has a faulty keyboard now,
	and all lowercase characters except those in \enquote{\ttt{mad printf}} (\enquote{a}, \enquote{d}, \enquote{f}, \enquote{i}, \enquote{m},
	\enquote{n}, \enquote{p}, \enquote{r}, and \enquote{t}) cannot be used.

	\subsection{Solution}

		Given that all the looping constructs (\ttt{for}, \ttt{while}, and even \ttt{goto}) cannot be used due to having illegal characters,
		the only logical solution is recursion.

		\begin{bulletlist}
			& It is not illegal to call \ttt{main} recursively
			& When called with no arguments, \ttt{argc} is \ttt{1}; so \ttt{main(i)} will initialise \ttt{i} to \ttt{1}
			& \ttt{1E4} is equally as valid as \ttt{1e4}, which is identical to \ttt{10000}
			& The first arm of the ternary operator can be omitted: \ttt{x?:y} is valid.
		\end{bulletlist}

		We came up with two independent solutions to this problem:

		\begin{listing}[!htbp]\begin{ccode}
			main(i){printf("%d,",i++);if(i<1E4)main(i);}
			\end{ccode}
			\caption{The solution for Count 2: Wildness}
		\end{listing}

		\begin{listing}[!htbp]\begin{ccode}
			i;main(){printf("%d,",++i);i>=9999?:main();}
			\end{ccode}
			\caption{An alternative solution for Count 2: Wildness}
		\end{listing}

		In both cases, the code was exactly \num{44} characters long.

	% end subsection

	\subsection{Flag}

		The flag was \cddcflag{Main\_might\_be\_just\_a\_function\_but\_it\_is\_really\_special!}.

	% end subsection

% end section















